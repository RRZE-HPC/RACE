% SIAM Shared Information Template
% This is information that is shared between the main document and any
% supplement. If no supplement is required, then this information can
% be included directly in the main document.



% Improving sparse matrix-vector multiply on cache architectures
Sparse matrix-vector multiplication (SpMV) is an integral part of many scientific algorithms. It is a bandwidth-limited operation and on cache-based architectures the main factors that influence performance are spatial locality in accessing the matrix, and temporal locality in re-using the elements of the vector. To address this problem over the last two decades a plethora of coloring and partitioning techniques and data structures to improve SpMV multiplication on cache-based architectures have been suggested including greedy graph coloring techniques, cache-oblivious methods, and hypergraph partitioning. One of the first studies, \eg to improve temporal locality was done, \eg by Toledo~\cite{Toledo:1997:IMP:279511.279532} who performed an extensive study of Cuthill--McKee (CM) {\CA Is it en dash } ordering techniques on three--dimensional finite-element test matrices when used in combination with blocking into small dense blocks. Various authors~\cite{Buluc:2011:RMA:2058524.2059503,Williams:2009:OSM:1513001.1513318,doi:10.1177/1094342004041296} used techniques such as cache blocking for symmetric SpMV multiplication by splitting the matrix into several smaller $p \times q$ sparse submatrices and presented an analytic cache-aware model to determine the optimal block size. Their algorithms and software are included in OSKI~\cite{1742-6596-16-1-071} which is a collection of low-level primitives of tuned sparse kernels for modern cache-based superscalar machines. Xing \etal \cite{Liu:2013:ESM:2464996.2465013} used similar techniques to improve SIMD efficiency on Intel Knights Landing architecture
and compared the performance againt various other many-core architectures such as NVIDIA Tesla K20X. More recent work can be found, \eg in~\cite{Buluc:2011:RMA:2058524.2059503,Liu:2015:CES:2751205.2751209}

Previous work on parallel SpMV multiplication has also focused on reducing communication volume in a distributed memory setting, often by using variants of graph or hypergraph partitioning techniques~\cite{Catalyurek:1999}. Yzelman and Bisseling \cite{doi:10.1137/080733243,Yzelman-thesis-2011} extended hypergraph partitioning techniques by a cache-oblivious method by permuting rows and columns of the input matrix using a recursive hypergraph-based sparse matrix partitioning scheme so that the resulting matrix induces cache-friendly behavior during the SpMV. Multicoloring is another most popular approach used in this field~\cite{MC}, but is sometimes not efficient on modern cache-based processors. There have been studies carried on to increase the efficiency of \acrfull{MC} {\CA should the hyphen be omitted??} and improving the heuristics, an overview of the methods can be found in~\cite{dist_k_def,COLPACK,equitable_color}. One of the most successful and effective methods in this regard is the \acrfull{ABMC}~\cite{ABMC} proposed by Iwashita \etal in 2012. In many applications it is important to compute a coloring with few colors in near-linear time \cite{doi:10.1137/13093426X}. In parallel methods, the optimistic (speculative) coloring method by Gebremedhin and Manne \cite{gebremedhin2000scalable} is the preferred approach~\cite{Boman:2016}. 
Until recently, only a few of these coloring and partitioning technique concepts made it into mainstream software, but the increasingly stringent performance requirements for fast SpMV multiplication have resulted in a number of recent implementations that have adopted these concepts, namely, the \acrshort{MKL} library~\cite{MKL} and node-level programming and performance primitives from the Trilinos’ Kokkos node package~\cite{kokkos}. A distributed memory coloring framework for distance-1\cite{BOZDAG2008515} and distance-2~\cite{doi:10.1137/080732158} coloring have been implemented in Zoltan. The framework provides efficient implementation for greedy graph coloring algorithms and it provides parallel dynamic load balancing and related services for a wide variety of applications, including finite element methods and matrix operations.


% Coloring and iterative solvers
Coloring techniques have also been extensively used within parallel Krylov subspace methods. One of the earliest works on parallelizing kernels within iterative methods having loop-carried dependencies is the red-black Gauss--Seidel scheme~\cite{RBGS}. Later Kamath and Sameh introduced a two-block partitioning scheme for parallelizing the Kaczmarz method on tridiagonal structures~\cite{Kamath}. A first general study on the convergence of these parallel methods was done early in 1980 by Elfving~\cite{Elfving1980}. Another line of research focuses on parallelizing dependent kernels while maintaining the same convergence behavior of sequential execution. One of the earliest known works in this category is the hyperplane method~\cite{saad} on FDM (finite difference method) like matrices. Extensions to this approach is given in ~\cite{cm-rcm} where a hybrid approach between \acrshort{MC} and the hyperplane method is used. However, the most general method which falls into this category is level-scheduling{\CA should I remove hyphen?}~\cite{saad}.  Efficient implementation of this method on triangular solvers can be attributed to Park \etal  ~\cite{park_ls}. Most of the above mentioned methods have been tested only for their applicability to parallelize \DONE dependent kernels and some of them are not capable of dealing with dependencies like \DTWO. The research on parallelizing \DONE dependent kernels has been strongly accelerated after the introduction of the HPCG benchmark~\cite{hpcg}. When it comes to \DTWO kernels, popular methods seen in the literature are locking-based methods, thread private local vectors~\cite{sparseX,thread_private_symm_spmv} for kernels like symmetric sparse matrix vector {\CA should it be vectors?} or with the usage of specially tailored sparse matrix data formats like compressed sparse blocks (CSB)~\cite{CSB} or \acrfull{RSB}~\cite{RSB}.


In this paper we present a range of implementation issues for present a novel recursive algebraic coloring approach solving general distance-$k$ dependencies and demonstrate that it has more parallelism and bandwith performance.

