\begin{itemize}
	\item In this paper we propose a method to parallelize sparse linear algebra kernels having dependencies. The method is called \RACE and is a recursive level-based approach.  
	
	\item The main aim of the paper is to explain the \RACE method and procedures involved in it. Furthermore a brief parameter study and performance modeling of kernels executed with \RACE is carried out.
	
	\item In this paper we test our method for both \DONE (\GS) and \DTWO (\SymmSpmv and \KACZ) dependencies, and analyze both exact (\SymmSpmv) and iterative kernels (\GS, \KACZ).
	
	\item For \DONE kernels we compare against the existing approaches of \MC \cite{MC}, \ABMC \cite{ABMC} and \MKL \cite{MKL} implementations.
	
	\item For \DTWO kernels along with comparing against existing approaches of \MC \cite{feast_mc} and implementations in \MKL \cite{MKL}, we also extend the \ABMC method for \DTWO coloring. To our knowledge this is the first paper which uses \ABMC method for \DTWO coloring.
	
	\item Comparisons with \SymmSpmv (an exact kernel) enables us to solely study the performance behavior of \RACE compared with others, since here effects like convergence does not play a role. Overall we achieve a speed-up of $2-2.5 \times$ compared to \MC and \MKL implementations, while we are on par with \ABMC for small matrices and for large matrices we gain almost a factor of $1.5-2 \times$ benefit. 
	
	\item Comparisons with iterative kernels also puts into light the convergence behavior of the method. Results show that the convergence of \RACE is better than \MC and is competitive with \ABMC method.
	
	\item Finally we compare our method against a different sparse matrix data format known as \RSB, which is a tailored data format for working on kernels having dependencies. How much details on this comparison should we include in the paper? 
\end{itemize}