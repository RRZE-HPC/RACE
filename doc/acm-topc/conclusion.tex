In this paper we have developed \acrshort{RACE}, a coloring algorithm and open-source library
implementation for exploiting parallelism in algorithms with inherent dependencies.
\acrshort{RACE} generates hardware-efficient \DK colorings of undirected graphs and puts 
emphasis on data access locality, load balancing, and 
parallelism that is adapted to the number of cores of the underlying architecture.  We
demonstrated these benefits by applying \acrshort{RACE} to \acrfull{SymmSpMV} on modern
multicore architectures and compared its performance against
standard multicoloring, algebraic block multicoloring, and \acrshort{MKL}
implementations. Average and maximum speedups of 1.4 and 2, respectively,
could be observed across a representative set of 31 matrices on
two modern Intel processors. 
%Motivated by the shortcomings of existing \acrshort{MC}
%methods in terms of hardware efficiency and to address node-level performance,
%we derived a novel recursive algebraic coloring algorithm which eliminates the
%shortcomings of previously existing coloring methods.
Our entire experimental and performance analysis process was backed by the
Roof{}line performance model, corroborating the optimality of
the \acrshort{RACE} approach in terms of resource utilization and shedding some new
light on the challenges of the \acrshort{SymmSpMV} kernel on modern hardware.
We demonstrated that \acrshort{RACE} runs very close to the Roof{}line limit for
most of the 31 test cases. Outliers were analyzed and discussed in detail.
%\acrshort{RACE} fully takes into account  deep memory hierarchies thus 
%hampering scalability and full-chip performance.

Similar to other \acrshort{MC} approaches, the \acrshort{RACE} method is not
limited to the \acrshort{SymmSpMV} kernel and can be used to efficiently
parallelize solvers and kernels having general \DK dependencies. Moreover, due
to the level-based formulation of \acrshort{RACE}, the framework has an added
advantage that allows us to address other classes of problems. Future work
with \acrshort{RACE} will involve variants of linear solvers and kernel
operations like in-place matrix powers and polynomials, which are of high
interest in the scientific community.



