In this paper we described coloring algorithms and the 
open-source library implementation \acrshort{RACE} for efficient \acrfull{SymmSpMV} on
modern multicore  architectures. Motivated by the shortcomings of existing \acrshort{MC} methods in  terms of hardware efficiency 
and to address node-level performance, we derived a novel recursive algebraic  coloring algorithm 
which eliminates the shortcomings of previously existing coloring methods.
\acrshort{RACE} generates hardware efficient \DK colorings of undirected graphs and puts 
emphasis on achieving data locality, load balancing, and sufficient levels of parallelism that matches the 
number of cores  of the underlying architecture. 
We demonstrated these benefits by comparing \acrshort{RACE} performance against other state-of-the art
coloring methods and  \acrshort{MKL} implementations. Futhermore, our entire performance experiments
was backed by a strong performance model,  which ensured us the optimality of the \acrshort{RACE} approach
and threw light into the modern challenges  of the \acrshort{SymmSpMV} kernel.
A comparison with a state-of-the-art library  supplied by the Intel vendor, using 31 sparse matrices 
on the latest Intel  processors, shows that the proposed approach in \acrshort{RACE}
obtains performance very close to the estimated prediction using our derived roofline performance model for 
\acrfull{SymmSpMV}. \acrshort{RACE} fully takes into account  deep memory hierarchies thus 
hampering scalability and full-chip performance.
 We obtained an average speedup of over 1.4 for \acrshort{SymmSpMV} compared to current 
state-of-the-art  coloring algorithm and can achieve speedups of up to 2.




However, similar to any other \acrshort{MC} approaches \acrshort{RACE} method 
is not just limited to the \acrshort{SymmSpMV} kernel
and can be used to efficiently parallelize solvers and kernels 
having general \DK dependencies. Moreover, due to the level-based formulation
of \acrshort{RACE} the framework has an added advantage to address 
other classes of problems. Future work with \acrshort{RACE} will involve the
kernel operations like in-place matrix powers and polynomials, which are of 
high interest in the scientific community. 



