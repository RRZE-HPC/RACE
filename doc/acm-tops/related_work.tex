{\GW Versuch einiger einleitender Worte um die beiden Themen kurz in Kontext zu setzen}
The efficient solution of linear systems or eigenvalue problems involving large sparse matrices has been an active research field in parallel and high performance computing for many decades. Well-known, traditional application areas include quantum physics, quantum chemistry or engineering. In recent years new fields such as social graph analysis or unsupervised machine learning have further increased the need for hardware-efficient, parallel sparse solvers. Assuming sufficiently large problems, the solvers are typically based on iterative subspace method and may include advanced preconditioning techniques. In many methods two components,  \Acrfull{SpMV}  and coloring techniques, are among the most crucial parts for hardware efficiency and parallel scalability. Typically, these two components are considered to be orthogonal, i.e. hardware efficiency for \Acrshort{SpMV} is mainly related to data formats and local structures while coloring is used to address dependencies in the enclosing iteration scheme.  Interestingly, the hardware efficient parallelization of symmetric \Acrshort{SpMV} has not attracted a lot of attention over the years, though symmetry is widespread in the application fields. 

The \Acrshort{SpMV} operation  is an essential building block 
in a number of applications such as algebraic multigrid methods, 
sparse iterative solvers, shortest path algorithms, breadth first search algorithms, 
and Markov cluster algorithms and therefore it is an integral part 
of many scientific algorithms.
In the past decades, much research has been
focusing on designing new data structures, efficient algorithms, and
parallelization  techniques for this sparse basic linear algebra
subprogram.  \acrshort{SpMV} is typically a main memory bandwidth limited operation and, on
cache-based architectures, the main factors that influence performance
are spatial locality in accessing the matrix, and temporal locality in
reusing the elements of the vector. To address this problem, over the
last two decades a plethora of  partitioning techniques
and data structures to improve \acrshort{SpMV} multiplication on cache-based
architectures have been suggested including %greedy graph coloring techniques, 
cache-oblivious methods, and hypergraph partitioning. One
of the first studies, \eg to improve temporal locality was done, \eg
by Toledo~\cite{Toledo:1997:IMP:279511.279532} who performed an
extensive study of Cuthill--McKee (CM) ordering
techniques on three--dimensional finite-element test matrices when
used in combination with blocking into small dense blocks. Various
authors~\cite{Williams:2009:OSM:1513001.1513318,doi:10.1177/1094342004041296}
used advanced data storage formats and techniques such as register and cache blocking  for  
\acrshort{SpMV} by splitting the matrix into several smaller $p \times
q$ sparse submatrices and presented an analytic cache-aware model to
determine the optimal block size. These algorithms are, e.g.,
included in OSKI~\cite{1742-6596-16-1-071} which is a collection of
low-level primitives of tuned sparse kernels for modern cache-based
superscalar machines. Kreutzer \etal~\cite{Moritz_sell} and 
Xing \etal~\cite{Liu:2013:ESM:2464996.2465013} used techniques to 
improve SIMD efficiency and performance on many-core and GPU architectures. 
Recent work can be found, \eg
in~\cite{li2017hbm,Liu:2015:CES:2751205.2751209,liu2015spmv}.
 Previous work on  \acrshort{SpMV}  has also focused on reducing
 communication volume in a distributed memory setting, often by using
 variants of graph or hypergraph partitioning
 techniques~\cite{Catalyurek:1999}. Yzelman and
 Bisseling~\cite{doi:10.1137/080733243,Yzelman-thesis-2011} extended
 hypergraph partitioning techniques by a cache-oblivious method by
 permuting rows and columns of the input matrix using a recursive
 hypergraph-based sparse matrix partitioning scheme so that the
 resulting matrix induces cache-friendly behavior during the
 \acrshort{SpMV}.

\begin{itemize}
	\item More linking from \acrshort{SpMV} to \acrshort{SymmSpMV}
	\item Many have concentrated on general \acrshort{SpMV}, with methods exploiting
	 structure in matrix like blocking, tuning and using tailored data storage formats and so on.
	 \item Performance modeling of general \acrshort{SpMV} is extensively carried out,
	 but there is no such work on \acrshort{SymmSpMV}.
\end{itemize}
%More recently Cheshmi~\etal~\cite{Cheshmi:2018:PIT:3291656.3291739} used e.g. a
%novel task coarsening strategy to create well-balanced tasks that can
%execute in parallel, while maintaining locality of memory accesses. 


%\cite{Buluc:2011:RMA:2058524.2059503

Despite \acrshort{SpMV} being a bandwidth-limited operation not much work has 
been done to exploit the symmetry property of symmetric matrices to reduce
storage requirements and data transfers by using only the upper/lower triangular part the matrix.
% and hence  optimizing storage and data transfer of symmetric matrices .
The major challenge here is to resolve the potential write conflicts of explicit \acrshort{SymmSpMV} kernels in parallel processing.
%is mainly due to the dependency problems that arise with the parallelization of the  \acrshort{SymmSpMV} operation. 
There are general solutions for such problems like 
lock based methods and thread private target
arrays~\cite{sparseX,thread_private_symm_spmv,Krotkiewski:2010:PSS:1752612.1752682,Mironowicz:2015}. However they have in common that their overhead may increase with degree of parallelism.
%which may potentially increase overhead with the level of parallelism.
Another recent research direction is the use of specialized storage formats 
like CSB \cite{CSB}, RSB \cite{RSB}, CSX \cite{sparseX} combined with the use of bitmasked 
register blocking techniques as in \cite{Buluc:2011:RMA:2058524.2059503}. As pointed out 
by \cite{liu2015spmv} this approach has severe drawbacks like missing backward compatibility or frequent conversion costs.  
Resolving the inherent data dependency in the \acrshort{SymmSpMV} kernel using a \DTWO coloring of the underlying undirected graph is another potential way of tackling the problem which has not been investigated so far to the best of our knowledge.

\begin{itemize}
	\item Another point in favor of coloring is most of the above approaches would work only for \acrshort{SymmSpMV} kernels and are not suitable for D-2 solvers like KACZ.
\end{itemize}

\Acrfull{MC} reordering to tackle data dependencies is a very well established strategy in parallelization of iterative solvers. As it is applied to the underlying graph it is not bound to a specific data format and may use existing highly optimized (serial) kernels, i.e. it is orthogonal to general code optimization strategies.  
Prominent examples for  \acrshort{MC} in iterative solvers are Gauss-Seidel, incomplete 
Cholesky factorization, Kaczmarz method~\cite{RBGS,MC,feast_mc} where typically a \DONE or \DTWO coloring is applied subject to the underlying dependencies of the iterative scheme. However, coloring changes the evaluation order of the original solver and may lead to worse convergence rates. This is different in the context of \acrshort{SymmSpMV} where we only need to ensure that a single entry of the target vector is not written in parallel and do not otherwise require strict serial ordering to get to the same result. 
In terms of hardware efficiency long-standing \acrshort{MC} methods often generate colorings which lack efficiency on modern cache-based
processors. Studies have been done to increase their
efficiency and improve inherent heuristics; an overview of the methods can be found
in~\cite{dist_k_def,COLPACK,equitable_color}. However, 
for irregular and/or large sparse 
matrices \acrshort{MC} may lead to load imbalance, frequent global synchronization, 
and loss of data locality, severely reducing (single-node) performance. 
These problems typically become more stringent for higher order distance
colorings and larger matrices.
One of the most
successful and effective methods in this regard is
the \acrfull{ABMC}~\cite{ABMC} proposed by Iwashita \etal in 2012, which tries to increase data locality by applying graph partitioning (blocking) before coloring. 
Widely used and public available coloring packages such as COLPACK\cite{COLPACK} and ZOLTAN\cite{BOZDAG2008515,doi:10.1137/080732158} also aim to speed-up the coloring process by parallelization and other heuristics. 

{\GW Olaf please check: Beziehen sich diese beiden Saetze un Referenzen auf den COloring Prozess? Falls ja - benoetigen wir diese? Dazu sagen wir ja gar nichts}
In many applications it is important to compute a coloring with few
colors in near-linear time~\cite{doi:10.1137/13093426X}.
In parallel coloring methods, the optimistic (speculative) coloring method by Gebremedhin
and Manne~\cite{gebremedhin2000scalable} is the preferred
approach~\cite{Boman:2016}.  
{\GW Olaf please check: Der folgende Satz steht meiner Meinung nach im Widerspruch zu unserer Aussage, dass noch niemand Coloring fuer Parallelisierung von SymmSpMV gemacht hat - siehe oben. Oder hast du da andere Informationen.}
Until recently, only a few of these
coloring and partitioning technique concepts made it into mainstream
software, but the increasingly stringent performance requirements for
fast \acrshort{SymmSpMV} multiplication have resulted in a number of recent
implementations that have adopted these concepts, namely,
{\GW Olaf please check: Meinst du dass MKL und KOKKOS primitives fuers Coloring oder fuer SymSpMV haben? Lt Christie hat MKL SymmSpMV Routinen aber KOKKOS nicht und beim Coloring ist es genau anders rum.}
the \acrshort{MKL} library~\cite{MKL} and node-level programming and
performance primitives from the Trilinos' Kokkos node
package~\cite{kokkos}. 
{\GW Olaf please check: Ist es OK wenn wir die folgenden Saetze mit dem letzten Satz des vorherigen Abschnittes abhandeln?}
A distributed memory coloring framework for
distance-1~\cite{BOZDAG2008515} and
distance-2~\cite{doi:10.1137/080732158} coloring have been implemented
in Zoltan. The framework provides efficient implementation for greedy
graph coloring algorithms and it provides parallel dynamic load
balancing and related services for a wide variety of applications,
including finite-element methods and matrix operations.  
%
{\GW Olaf please check: Das wuerde ich allgemeiner formulieren: Tebdenz: Mehr Threads; hoehere Bandbreite aber auch uU kleinere Memories (HBM etc) --> explizite hw-effiziente SymmSPMV kernels inreasingly important.}
However, it has been very clear, for some time, that in order to provide better
support of \acrshort{SymmSpMV} running on the next generation of
computing platforms, fundamental problems related to model
representation and coloring techniques on modern hardware accelerated
computational platforms has to be revisted again. One needs to think
deeper about the ramifications to the entire software stack so
that one can transition the algorithms to once again realign with the
realities of the underlying hardware constraints. 

\CAcomm{Currently I have not included Cheshmi~\etal~\cite{Cheshmi:2018:PIT:3291656.3291739}  as it 
	does not fit to the text.}

{\GW Olaf please check: Das ist eine Zusammenfassung der Methode - die passt nicht in Einleitung/Related Work; bestenfalls in die Contributions.}
In this paper we present a novel recursive algebraic coloring approach solving general distance-$k$ dependencies and demonstrate that it has more parallelism and bandwith performance. It is motivated by the shortcomings of existing \acrshort{MC} methods in terms of hardware efficiency and parallelization overhead. Our method addresses matrices that can be represented by an undirected graph. In a first step we do a \acrlong{BFS} preprocessing for bandwidth reduction of the graph, which aims to increase data locality for the underlying sparse matrix problems: Starting from a root vertex we construct level-set by using the \acrshort{BFS} algorithm \ie \level $i$ consists of all nodes having distance $i$ to the root vertex. We then permute the graph such that vertex numbering increases with distance from the root vertex. Coloring the resulting \levels would be a naive approach to generate a \DK coloring but would for, obvious reasons (\eg \level 0 contains only one vertex), often lead to severe load imbalance. Thus we perform in a second step $level$ $aggregation$ of neighboring \levels, which aims at conserving data locality. The choice of the size of each \levelGroup is subject to two major constraints: First, for a \DK coloring of the original graph/problem at least $k$ \levels are aggregated into a \levelGroup. This means that alternate \levelGroups can be executed in parallel which is equivalent to a \DONE coloring of the \levelGroups. Second, we apply a criterion for load balancing that considers the total amount of hardware threads to be used at execution time and tries to balance workload across these threads evenly. At this stage it might happen that most of the vertices end up in a few \levelGroups. Therefore, depending on the size of the \levelGroups, a different number of threads will be assigned to each of them. \Inorder to further parallelize within this \levelGroup for assigned threads the entire procedure is recursively repeated on their corresponding \subgraphs subject to the \DK constraint. The aggregation step is controlled by a single external parameter which influences the load imbalance introduced by forming each \levelGroup. Due to the recursive nature of this algorithm, nested parallelism is required. However, only local synchronization is required between the threads assigned to the same \subgraph.


